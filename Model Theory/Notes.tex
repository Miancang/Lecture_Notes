\documentclass[12pt, reqno]{amsart}
\usepackage{amsmath, amsthm, amscd, amsfonts, amssymb, graphicx, xcolor, tikz-cd}
\usepackage[bookmarksnumbered, colorlinks, plainpages]{hyperref}

\textheight 22.5truecm \textwidth 14.5truecm
\setlength{\oddsidemargin}{0.35in}\setlength{\evensidemargin}{0.35in}

\setlength{\topmargin}{-.5cm}

\newtheorem{theorem}{Theorem}[section]
\newtheorem{lemma}[theorem]{Lemma}
\newtheorem{proposition}[theorem]{Proposition}
\newtheorem{corollary}[theorem]{Corollary}
\theoremstyle{definition}
\newtheorem{definition}[theorem]{Definition}
\newtheorem{example}[theorem]{Example}
\newtheorem{exercise}[theorem]{Exercise}
\newtheorem{conclusion}[theorem]{Conclusion}
\newtheorem{conjecture}[theorem]{Conjecture}
\newtheorem{criterion}[theorem]{Criterion}
\newtheorem{summary}[theorem]{Summary}
\newtheorem{axiom}[theorem]{Axiom}
\newtheorem{problem}[theorem]{Problem}
\theoremstyle{remark}
\newtheorem{remark}[theorem]{Remark}
\numberwithin{equation}{section}

\begin{document}
\setcounter{page}{1}

\color{darkgray}{
\noindent 


\centerline{}

\centerline{}


\title[Notes on Model Theories]{Notes on Model Theories}


\author{Xingzhi Huang}

 \maketitle

\section{Structures}

\begin{definition}
    A structure $\mathcal{A}:= (A, \{R_i^{\mathcal{A}}\}_{i\in I}, \{f_j^{\mathcal{A}}\}_{j\in J}, \{c_k^{\mathcal{A}}\}_{k\in K})$ consists of: 
    
    1. a non-empty set $A$, called the universe of $\mathcal{A}$
    
    2. a relation $R_i^{\mathcal{A}}$ on $A$ for each $i\in I$
    
    3. a function $f_j^{\mathcal{A}}: A^{n_j}\to A$ for each $j\in J$

    4. an element $c_k^{\mathcal{A}}\in A$ for each $k\in K$.

    The sets $I, J, K$ are disjoint index sets. The arity of $R_i^{\mathcal{A}}$ is some natural number $m_i$, and the arity of $f_j^{\mathcal{A}}$ is some natural number $n_j$.
\end{definition}

\begin{definition}
    Given a structure $\mathcal{A}= (A, \{R_i^{\mathcal{A}}\}_{i\in I}, \{f_j^{\mathcal{A}}\}_{j\in J}, \{c_k^{\mathcal{A}}\}_{k\in K})$, the signature $L_\mathcal{A}$ of $\mathcal{A}$ is the collection of symbols $\{R_i\}_{i\in I}, \{f_j\}_{j\in J}, \{c_k\}_{k\in K}$, where $R_i$ is a relation symbol of arity $m_i$, $f_j$ is a function symbol of arity $n_j$, and $c_k$ is a constant symbol.
\end{definition}

\begin{definition}
    Fix a signature $L$. An $L$-structure is a structure whose signature is $L$.
\end{definition}

\begin{definition}
    We can define a category of $L$-structures, denoted by $Str(L)$, as follows.
    
    The objects of $Str(L)$ are all $L$-structures. If $\mathcal{A}$ and $\mathcal{B}$ are two $L$-structures, then a morphism from $\mathcal{A}$ to $\mathcal{B}$ is a function $h: A\to B$ such that:

    1. for each relation symbol $R_i$ of arity $m_i$ in $L$, and for all $a_1, a_2, \ldots, a_{m_i}\in A$, if $(a_1, a_2, \ldots, a_{m_i})\in R_i^{\mathcal{A}}$, then $(h(a_1), h(a_2), \ldots, h(a_{m_i}))\in R_i^{\mathcal{B}}$;

    2. for each function symbol $f_j$ of arity $n_j$ in $L$, and for all $a_1, a_2, \ldots, a_{n_j}\in A$, we have $h(f_j^{\mathcal{A}}(a_1, a_2, \ldots, a_{n_j})) = f_j^{\mathcal{B}}(h(a_1), h(a_2), \ldots, h(a_{n_j}))$;

    3. for each constant symbol $c_k$ in $L$, we have $h(c_k^{\mathcal{A}}) = c_k^{\mathcal{B}}$.
\end{definition}

\begin{theorem}
    The injective morphisms in $Str(L)$ are precisely the embeddings; the surjective morphisms in $Str(L)$ are precisely the surjective homomorphisms.
\end{theorem}

\begin{definition}
    Given two $L$-structures $\mathcal{A}$ and $\mathcal{B}$, we say that $\mathcal{A}$ is a substructure of $\mathcal{B}$, denoted by $\mathcal{A}\subseteq \mathcal{B}$, if $A\subseteq B$ and the inclusion map $i: A\to B$ is an embedding.
\end{definition}

\section{Terms and Formulas}

\subsection{Syntax}

\begin{definition}
    Let $L$ be a signature. The set of $L$-terms is defined inductively as follows:

    1. Every variable is an $L$-term.

    2. Every constant symbol in $L$ is an $L$-term.

    3. If $f$ is an $n$-ary function symbol in $L$ and $t_1, t_2, \ldots, t_n$ are $L$-terms, then $f(t_1, t_2, \ldots, t_n)$ is an $L$-term.

    4. Nothing else is an $L$-term.
\end{definition}

\begin{remark}
    A term is called closed if it contains no variables.
\end{remark}

\begin{remark}
    The complexity of a term $t$, denoted by $c(t)$, is defined inductively as follows:
    
    1. If $t$ is a variable or a constant symbol, then $c(t) = 0$.

    2. If $t$ is of the form $f(t_1, t_2, \ldots, t_n)$, where $f$ is an $n$-ary function symbol and $t_1, t_2, \ldots, t_n$ are $L$-terms, then $c(t) = 1 + \max(c(t_1), c(t_2), \ldots, c(t_n))$.
\end{remark}

\begin{definition}
    Let $L$ be a signature. The set of atomic $L$-formulas is defined as follows:

    1. If $t_1$ and $t_2$ are $L$-terms, then $(t_1 = t_2)$ is an atomic $L$-formula.

    2. If $R$ is an $n$-ary relation symbol in $L$ and $t_1, t_2, \ldots, t_n$ are $L$-terms, then $R(t_1, t_2, \ldots, t_n)$ is an atomic $L$-formula.

    3. Nothing else is an atomic $L$-formula.
\end{definition}

\subsection{Semantics}

\begin{definition}
    Let $t(x_1, x_2, \ldots, x_n)$ be an $L$-term with variables among $x_1, x_2, \ldots, x_n$. Let $\mathcal{A}$ be an $L$-structure and let $a_1, a_2, \ldots, a_n\in A$. The interpretation of $t$ in $\mathcal{A}$ at $(a_1, a_2, \ldots, a_n)$, denoted by $t^{\mathcal{A}}(a_1, a_2, \ldots, a_n)$, is defined inductively as follows:

    1. If $t$ is a variable $x$, then $t^{\mathcal{A}}(a) = a$.

    2. If $t$ is a constant symbol $c$, then $t^{\mathcal{A}}() = c^A$.(Here we use $c^A$ to indicate $c^A$ live in the semantics world.)

    3. If $t$ is of the form $f(t_1, t_2, \ldots, t_m)$, where $f$ is an $m$-ary function symbol and $t_1, t_2, \ldots, t_m$ are $L$-terms, then
    \[
        t^{\mathcal{A}}(a_1, a_2, \ldots, a_n) = f^{\mathcal{A}}(t_1^{\mathcal{A}}(a_1, a_2, \ldots, a_n), t_2^{\mathcal{A}}(a_1, a_2, \ldots, a_n), \ldots, t_m^{\mathcal{A}}(a_1, a_2, \ldots, a_n)).
    \]
    
\end{definition}

\begin{definition}
    Let $\varphi(x_1, x_2, \ldots, x_n)$ be an atomic $L$-formula with variables among $x_1, x_2, \ldots, x_n$. Let $\mathcal{A}$ be an $L$-structure and let $a_1, a_2, \ldots, a_n\in A$. We say that $\varphi$ is true in $\mathcal{A}$ at $(a_1, a_2, \ldots, a_n)$, or $(a_1, a_2, \ldots, a_n)$ satisfies $\varphi$ in $\mathcal{A}$, denoted by $\mathcal{A}\models \varphi(a_1, a_2, \ldots, a_n)$, if one of the following conditions holds:

    1. If $\varphi$ is of the form $(t_1 = t_2)$, where $t_1$ and $t_2$ are $L$-terms, then $\mathcal{A}\models \varphi(a_1, a_2, \ldots, a_n)$ if and only if $t_1^{\mathcal{A}}(a_1, a_2, \ldots, a_n) = t_2^{\mathcal{A}}(a_1, a_2, \ldots, a_n)$.

    2. If $\varphi$ is of the form $R(t_1, t_2, \ldots, t_m)$, where $R$ is an $m$-ary relation symbol and $t_1, t_2, \ldots, t_m$ are $L$-terms, then $\mathcal{A}\models \varphi(a_1, a_2, \ldots, a_n)$ if and only if $(t_1^{\mathcal{A}}(a_1, a_2, \ldots, a_n), t_2^{\mathcal{A}}(a_1, a_2, \ldots, a_n), \ldots, t_m^{\mathcal{A}}(a_1, a_2, \ldots, a_n))\in R^{\mathcal{A}}$.
\end{definition}

\begin{theorem}
    Let $A, B$ be two $L$-structures and let $f: A\to B$ be a morphism. Let $\varphi(x_1, x_2, \ldots, x_n)$ be an atomic $L$-formula with variables among $x_1, x_2, \ldots, x_n$. Then for all $a_1, a_2, \ldots, a_n\in A$, $f(t^{A}(a_1, a_2, \ldots, a_n)) = t^{B}(f(a_1), f(a_2), \ldots, f(a_n))$.
\end{theorem}

\begin{corollary}
    Let $A, B$ be two $L$-structures and let $f: A\to B$ be a morphism of $\mathbf{Set}$. Then:

    1. $f$ is a morphism of $Str(L)$ if and only if for every atomic $L$-formula $\varphi(x_1, x_2, \ldots, x_n)$ and all $a_1, a_2, \ldots, a_n\in A$,
   \[
    \mathcal{A}\models \varphi(a_1, a_2, \ldots, a_n) \implies \mathcal{B}\models \varphi(f(a_1), f(a_2), \ldots, f(a_n)).
   \]

    2. If $f$ is an embedding, then for every atomic $L$-formula $\varphi(x_1, x_2, \ldots, x_n)$ and all $a_1, a_2, \ldots, a_n\in A$,
    \[
    \mathcal{A}\models \varphi(a_1, a_2, \ldots, a_n) \iff \mathcal{B}\models \varphi(f(a_1), f(a_2), \ldots, f(a_n)).
    \]
\end{corollary}

\section{Canonical Models}

\begin{lemma}
    Let $\mathcal{A}$ and $\mathcal{B}$ be $L$-structures, and let $\bar{a} = (a_1, \ldots, a_n) \in A^n$, $\bar{b} = (b_1, \ldots, b_n) \in B^n$. Consider the expanded signatures $L(\bar{c})$ where $\bar{c} = (c_1, \ldots, c_n)$ are new constant symbols, and the $L(\bar{c})$-structures $(\mathcal{A}, \bar{a})$ and $(\mathcal{B}, \bar{b})$ interpreting $c_i$ as $a_i$ and $b_i$ respectively.

    The following are equivalent:
    \begin{enumerate}
        \item For every atomic $L(\bar{c})$-sentence $\varphi$, if $(\mathcal{A}, \bar{a}) \models \varphi$ then $(\mathcal{B}, \bar{b}) \models \varphi$.
        \item There exists a homomorphism $f: \langle \bar{a} \rangle_{\mathcal{A}} \to \mathcal{B}$ such that $f(a_i) = b_i$ for all $i$.
    \end{enumerate}
    Moreover, the homomorphism $f$ in (2) is unique if it exists, and $f$ is an embedding if and only if
    \[
        \text{for every atomic $L(\bar{c})$-sentence $\varphi$, } (\mathcal{A}, \bar{a}) \models \varphi \iff (\mathcal{B}, \bar{b}) \models \varphi.
    \]
\end{lemma}

\begin{definition}
    Let $L$ be a signature and $A$ be a $L$-structure. A set $T$ of atomic $L$-sentences is called a closed set of atomic $L$-sentences if the following conditions hold:

    1. For every closed term $t$, the sentence $(t = t)$ is in $T$.

    2. For every $n$-ary function symbol $f$ and closed terms $t_1, t_2, \ldots, t_n, s_1, s_2, \ldots, s_n$, if the sentences $(t_1 = s_1), (t_2 = s_2), \ldots, (t_n = s_n)$ and $(f(t_1, t_2, \ldots, t_n) = r)$ are in $T$, then the sentence $(f(s_1, s_2, \ldots, s_n) = r)$ is also in $T$.
\end{definition}

TBC

\section{First-Order Logic}

\subsection{Syntax}

\begin{definition}
    Let $L$ be a signature. The class of $L$-formulas is defined inductively as follows:

    1. Every atomic $L$-formula is an $L$-formula.

    2. If $\varphi$ is an $L$-formula, then $(\neg \varphi)$ is also an $L$-formula. If $\Phi$ is a set of atmoic $L$-formulas, then $(\bigwedge_{\varphi\in \Phi} \varphi)$ and $(\bigvee_{\varphi\in \Phi} \varphi)$ are also $L$-formulas.

    3. If $\varphi$ is an $L$-formula and $x$ is a variable, then $(\forall x \varphi)$ and $(\exists x \varphi)$ are also $L$-formulas.
\end{definition}

\subsection{Semantics}

\begin{definition}
    Let $A$ be an $L$-structure. The satisfaction relation $\mathcal{A} \models \varphi(a_1, a_2, \ldots, a_n)$ for an $L$-formula $\varphi(x_1, x_2, \ldots, x_n)$ and elements $a_1, a_2, \ldots, a_n \in A$ is defined inductively as follows:

    1. If $\varphi$ is an atomic $L$-formula, then $\mathcal{A} \models \varphi(a_1, a_2, \ldots, a_n)$ is defined as in the previous section.

    2. If $\varphi$ is of the form $(\neg \psi)$, then $\mathcal{A} \models \varphi(a_1, a_2, \ldots, a_n)$ if and only if $\mathcal{A} \not\models \psi(a_1, a_2, \ldots, a_n)$.

    3. If $\varphi$ is of the form $(\bigwedge_{\psi\in \Phi} \psi)$, where $\Phi$ is a set of $L$-formulas, then $\mathcal{A} \models \varphi(a_1, a_2, \ldots, a_n)$ if and only if for every $\psi \in \Phi$, $\mathcal{A} \models \psi(a_1, a_2, \ldots, a_n)$.

    4. If $\varphi$ is of the form $(\bigvee_{\psi\in \Phi} \psi)$, where $\Phi$ is a set of $L$-formulas, then $\mathcal{A} \models \varphi(a_1, a_2, \ldots, a_n)$ if and only if for some $\psi \in \Phi$, $\mathcal{A} \models \psi(a_1, a_2, \ldots, a_n)$.

    5. If $\varphi$ is of the form $(\forall x \psi)$, then $\mathcal{A} \models \varphi(a_1, a_2, \ldots, a_n)$ if and only if for every $b \in A$, $\mathcal{A} \models \psi(b, a_1, a_2, \ldots, a_n)$.

    6. If $\varphi$ is of the form $(\exists x \psi)$, then $\mathcal{A} \models \varphi(a_1, a_2, \ldots, a_n)$ if and only if there exists some $b \in A$ such that $\mathcal{A} \models \psi(b, a_1, a_2, \ldots, a_n)$.
\end{definition}

\begin{remark}
    We can define the (infinity) language $L_{\infty \omega}$ to be the collection of all $L$-formulas.
\end{remark}

\begin{definition}
    Let $L$ be a signature, the first-order language over $L$, denoted by $L_{\omega \omega}$, is the collection of all $L$-formulas where the conjunctions and disjunctions are finite.
\end{definition}

\begin{definition}
    Fix a signature $L$. A sentence is an $L$-formula with no free variables. A class $T$ of $L$-sentences is called a theory.
\end{definition}

\begin{definition}
    Let $L$ be a signature and $T$ be a theory of $L$. An $L$-structure $\mathcal{A}$ is called a model of $T$, denoted by $\mathcal{A} \models T$, if for every sentence $\varphi \in T$, we have $\mathcal{A} \models \varphi$.
\end{definition}

\begin{definition}
    Let $T$ be a theory in $L_{\infty \omega}$ and $K$ a class of $L$-structures. We say that $T$ axiomatizes $K$ if for every $L$-structure $\mathcal{A}$, $\mathcal{A} \in K$ if and only if $\mathcal{A} \models T$.
\end{definition}

\section{Compactness Theorem}

\subsection{Teminology}

\begin{definition}
    A theory $T$ in $L_{\infty \omega}$ is called a Hintikka set if it satisfies the following conditions:

    1. For every closed term $t$, the sentence $(t = t)$ is in $T$.

    2. If $\varphi \in T$, then $(\neg \varphi) \in T$.

    3. If $\varphi$ is an atmoic formula in $T$, $s,t$ are closed terms, and $(s = t) \in T$, then the $\varphi(s)$ is in $T$ if and only if $\varphi(t)$ is in $T$.

    4. If $(\neg \neg \varphi) \in T$, then $\varphi \in T$.

    5. If $(\bigwedge_{\varphi\in \Phi} \varphi) \in T$, then for every $\varphi \in \Phi$, we have $\varphi \in T$; if $\neg(\bigwedge_{\varphi\in \Phi} \varphi) \in T$, then there exists some $\varphi \in \Phi$ such that $(\neg \varphi) \in T$.

    6. If $(\bigvee_{\varphi\in \Phi} \varphi) \in T$, then there exists some $\varphi \in \Phi$ such that $\varphi \in T$; if $\neg(\bigvee_{\varphi\in \Phi} \varphi) \in T$, then for every $\varphi \in \Phi$, we have $(\neg \varphi) \in T$.

    7. If $(\forall x \varphi) \in T$, then for every closed term $t$, we have $\varphi(t) \in T$; if $(\neg \forall x \varphi) \in T$, then there exists some closed term $t$ such that $(\neg \varphi(t)) \in T$.

    8. If $(\exists x \varphi) \in T$, then there exists some closed term $t$ such that $\varphi(t) \in T$; if $(\neg \exists x \varphi) \in T$, then for every closed term $t$, we have $(\neg \varphi(t)) \in T$.
\end{definition}

\begin{example}
    Let $T$ be the class of all $L$-sentences that are true in some $L$-structure $\mathcal{A}$. Then $T$ is a Hintikka set.
\end{example}

\begin{theorem}
    Let $T$ be a Hintikka set in $L_{\omega \omega}$. Then there exists an $L$-structure $\mathcal{A}$ such that $\mathcal{A} \models T$.
\end{theorem}

\begin{theorem}
    Let $L$ be a first-order language and $T$ be a theory in $L$. $T$ is a Hintikka set if it satisfies the following conditions:

    1. Every finite subset of $T$ has a model;

    2. For every sentence $\varphi \in L$, either $\varphi \in T$ or $(\neg \varphi) \in T$.
    
    3. For every sentence $\exists x \varphi(x) \in T$, there exists some closed term $t\in L$ such that $\varphi(t) \in T$.


\end{theorem}
\subsection{Main Theorems}

\begin{theorem}[Compactness Theorem]
    Let $L$ be a first-order language and $T$ be a theory in $L$. If every finite subset of $T$ has a model, then $T$ has a model.
\end{theorem}

\subsection{Filters and Ultrafilters}

\begin{definition}
    Given a set $S$, a filter $\mathcal{F}$ is a subset of $\mathcal{P}(S)$ such that:

    1. $\emptyset \notin \mathcal{F}$;

    2. if $A, B \in \mathcal{F}$, then $A \cap B \in \mathcal{F}$;

    3. if $A \in \mathcal{F}$ and $A \subseteq B \subseteq S$, then $B \in \mathcal{F}$.
\end{definition}

\begin{definition}
    A filter $\mathcal{U}$ on a set $S$ is called an ultrafilter if for every $A \subseteq S$, either $A \in \mathcal{U}$ or $S \setminus A \in \mathcal{U}$.
\end{definition}

\begin{theorem}
    If $\mathcal{F}$ is a filter on a set $S$, $F_1, F_2, \ldots, F_n$ are subsets of $S$, then $F_1 \cap F_2 \cap \ldots \cap F_n \neq \emptyset$.
\end{theorem}

\begin{theorem}
    Let $S$ be a set and $\mathcal{F}\subseteq \mathcal{P}(S)$. If $\mathcal{F}$ has the finite intersection property, i.e., for every finite subset $\{F_1, F_2, \ldots, F_n\} \subseteq \mathcal{F}$, we have $F_1 \cap F_2 \cap \ldots \cap F_n \neq \emptyset$, then there exists an ultrafilter $\mathcal{U}$ on $S$ such that $\mathcal{F} \subseteq \mathcal{U}$.
\end{theorem}
\begin{proof}
    Notice that given a subset of $\mathcal{P}(S)$ which satisfies the finite intersection property, we can find a filter containing it. Furthermore, we can add $X$ or $S\setminus X$ for any $X\subseteq S$ to the subset so that the finite intersection property still holds. By Zorn's lemma(?Maybe transfinite induction, i have no idea why zorn's lemma fails), we can find a maximal such subset, which gives rise to an ultrafilter (Actually if a subset satisfying finite intersection property and contains every one of the pair of the complements, then itself is an ultrafilter).
\end{proof}

\begin{definition}
    Given a collection of sets $M_x$ indexed by a set $X$ and an ultrafilter $\mathcal{U}$ on $X$, the ultraproduct $\prod_{x \in X} M_x/ \mathcal{U}$ is defined as the quotient of the Cartesian product $\prod_{x \in X} M_x$ by the equivalence relation $\sim$ defined as follows: for $(m_x), (n_x) \in \prod_{x \in X} M_x$, we have $(m_x) \sim (n_x)$ if and only if $\{x \in X : m_x = n_x\} \in \mathcal{U}$.
\end{definition}



\begin{definition}
    Given a collection of structures $\mathcal{A}_x$ indexed by a set $X$ and an ultrafilter $\mathcal{U}$ on $X$, we can define the ultraproduct $\prod_{x \in X} \mathcal{A}_x/ \mathcal{U}$ as follows:

    1. The universe of $\prod_{x \in X} \mathcal{A}_x/ \mathcal{U}$ is the ultraproduct of the universes of $\mathcal{A}_x$, i.e., $\prod_{x \in X} A_x/ \mathcal{U}$.

    2. For each $n$-ary relation symbol $R$ in the signature, we define $R^{\prod_{x \in X} \mathcal{A}_x/ \mathcal{U}}$ as follows: for $[(a_x^1)], [(a_x^2)], \ldots, [(a_x^n)] \in \prod_{x \in X} A_x/ \mathcal{U}$, we have
    \[
        ([(a_x^1)], [(a_x^2)], \ldots, [(a_x^n)]) \in R^{\prod_{x \in X} \mathcal{A}_x/ \mathcal{U}} \iff \{x \in X : (a_x^1, a_x^2, \ldots, a_x^n) \in R^{\mathcal{A}_x}\} \in \mathcal{U}.
    \]

    3. For each $n$-ary function symbol $f$ in the signature, we define $f^{\prod_{x \in X} \mathcal{A}_x/ \mathcal{U}}$ as follows: for $[(a_x^1)], [(a_x^2)], \ldots, [(a_x^n)] \in \prod_{x \in X} A_x/ \mathcal{U}$, we have
    \[
        f^{\prod_{x \in X} \mathcal{A}_x/ \mathcal{U}}([(a_x^1)], [(a_x^2)], \ldots, [(a_x^n)]) = [(b_x)],
    \]
    where $b_x = f^{\mathcal{A}_x}(a_x^1, a_x^2, \ldots, a_x^n)$ for each $x \in X$.

    4. For each constant symbol $c$ in the signature, we define $c^{\prod_{x \in X} \mathcal{A}_x/ \mathcal{U}} = [(c^{\mathcal{A}_x})]$.
\end{definition}

\begin{theorem}[Łoś's Theorem]
    Let $\{\mathcal{A}_x : x \in X\}$ be a collection of $L$-structures and $\mathcal{U}$ be an ultrafilter on $X$. For any $L$-formula $\varphi(x_1, x_2, \ldots, x_n)$ and elements $[(a_x^1)], [(a_x^2)], \ldots, [(a_x^n)] \in \prod_{x \in X} A_x/ \mathcal{U}$, we have
    \[
        \prod_{x \in X} \mathcal{A}_x/ \mathcal{U} \models \varphi([(a_x^1)], [(a_x^2)], \ldots, [(a_x^n)]) \iff \{x \in X : \mathcal{A}_x \models \varphi(a_x^1, a_x^2, \ldots, a_x^n)\} \in \mathcal{U}.
    \]

\end{theorem}

\begin{theorem}[Tarski-Vaught Test]
    Let $\mathcal{A}$ and $\mathcal{B}$ be $L$-structures with $\mathcal{A} \subseteq \mathcal{B}$. Then $\mathcal{A}$ is an elementary substructure of $\mathcal{B}$, denoted by $\mathcal{A} \preceq \mathcal{B}$, if and only if for every $L$-formula $\varphi(x, y_1, y_2, \ldots, y_n)$ and all $a_1, a_2, \ldots, a_n \in A$, whenever $\mathcal{B} \models \exists x \varphi(x, a_1, a_2, \ldots, a_n)$, there exists some $b \in A$ such that $\mathcal{B} \models \varphi(b, a_1, a_2, \ldots, a_n)$.
    
    
\end{theorem}

\subsection{Proof}

\begin{proof}
    Given $\varphi \in T$, we can define $T_\varphi = \{\text{finite subsets of } T \text{ containing } \varphi \text{ (that have a model)}\}$, $\mathcal{F} = \{T_\varphi: \varphi \in T\}$. It is easy to see that $\mathcal{F}$ has the finite intersection property. By the previous theorem, there exists an ultrafilter $\mathcal{U}$ on $T$ such that $\mathcal{F} \subseteq \mathcal{U}$.
\end{proof}





\bibliographystyle{amsplain}
\begin{thebibliography}{99}
\end{thebibliography}

\end{document}I