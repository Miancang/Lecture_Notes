\documentclass[12pt, reqno]{amsart}
\usepackage{amsmath, amsthm, amscd, amsfonts, amssymb, graphicx, xcolor, tikz-cd}
\usepackage[bookmarksnumbered, colorlinks, plainpages]{hyperref}

\textheight 22.5truecm \textwidth 14.5truecm
\setlength{\oddsidemargin}{0.35in}\setlength{\evensidemargin}{0.35in}

\setlength{\topmargin}{-.5cm}

\newtheorem{theorem}{Theorem}[section]
\newtheorem{lemma}[theorem]{Lemma}
\newtheorem{proposition}[theorem]{Proposition}
\newtheorem{corollary}[theorem]{Corollary}
\theoremstyle{definition}
\newtheorem{definition}[theorem]{Definition}
\newtheorem{example}[theorem]{Example}
\newtheorem{exercise}[theorem]{Exercise}
\newtheorem{conclusion}[theorem]{Conclusion}
\newtheorem{conjecture}[theorem]{Conjecture}
\newtheorem{criterion}[theorem]{Criterion}
\newtheorem{summary}[theorem]{Summary}
\newtheorem{axiom}[theorem]{Axiom}
\newtheorem{problem}[theorem]{Problem}
\theoremstyle{remark}
\newtheorem{remark}[theorem]{Remark}
\numberwithin{equation}{section}

\begin{document}
\setcounter{page}{1}

\color{darkgray}{
\noindent 


\centerline{}

\centerline{}


\title[Notes on Model Theories]{Notes on Model Theories}


\author{Xingzhi Huang}

 \maketitle

\section{Structures}

\begin{definition}
    A structure $\mathcal{A}:= (A, \{R_i^{\mathcal{A}}\}_{i\in I}, \{f_j^{\mathcal{A}}\}_{j\in J}, \{c_k^{\mathcal{A}}\}_{k\in K})$ consists of: 
    
    1. a non-empty set $A$, called the universe of $\mathcal{A}$
    
    2. a relation $R_i^{\mathcal{A}}$ on $A$ for each $i\in I$
    
    3. a function $f_j^{\mathcal{A}}: A^{n_j}\to A$ for each $j\in J$

    4. an element $c_k^{\mathcal{A}}\in A$ for each $k\in K$.

    The sets $I, J, K$ are disjoint index sets. The arity of $R_i^{\mathcal{A}}$ is some natural number $m_i$, and the arity of $f_j^{\mathcal{A}}$ is some natural number $n_j$.
\end{definition}

\begin{definition}
    Given a structure $\mathcal{A}= (A, \{R_i^{\mathcal{A}}\}_{i\in I}, \{f_j^{\mathcal{A}}\}_{j\in J}, \{c_k^{\mathcal{A}}\}_{k\in K})$, the signature $L_\mathcal{A}$ of $\mathcal{A}$ is the collection of symbols $\{R_i\}_{i\in I}, \{f_j\}_{j\in J}, \{c_k\}_{k\in K}$, where $R_i$ is a relation symbol of arity $m_i$, $f_j$ is a function symbol of arity $n_j$, and $c_k$ is a constant symbol.
\end{definition}

\begin{definition}
    Fix a signature $L$. An $L$-structure is a structure whose signature is $L$.
\end{definition}

\begin{definition}
    We can define a category of $L$-structures, denoted by $Str(L)$, as follows.
    
    The objects of $Str(L)$ are all $L$-structures. If $\mathcal{A}$ and $\mathcal{B}$ are two $L$-structures, then a morphism from $\mathcal{A}$ to $\mathcal{B}$ is a function $h: A\to B$ such that:

    1. for each relation symbol $R_i$ of arity $m_i$ in $L$, and for all $a_1, a_2, \ldots, a_{m_i}\in A$, if $(a_1, a_2, \ldots, a_{m_i})\in R_i^{\mathcal{A}}$, then $(h(a_1), h(a_2), \ldots, h(a_{m_i}))\in R_i^{\mathcal{B}}$;

    2. for each function symbol $f_j$ of arity $n_j$ in $L$, and for all $a_1, a_2, \ldots, a_{n_j}\in A$, we have $h(f_j^{\mathcal{A}}(a_1, a_2, \ldots, a_{n_j})) = f_j^{\mathcal{B}}(h(a_1), h(a_2), \ldots, h(a_{n_j}))$;

    3. for each constant symbol $c_k$ in $L$, we have $h(c_k^{\mathcal{A}}) = c_k^{\mathcal{B}}$.
\end{definition}

\begin{theorem}
    The injective morphisms in $Str(L)$ are precisely the embeddings; the surjective morphisms in $Str(L)$ are precisely the surjective homomorphisms.
\end{theorem}

\begin{definition}
    Given two $L$-structures $\mathcal{A}$ and $\mathcal{B}$, we say that $\mathcal{A}$ is a substructure of $\mathcal{B}$, denoted by $\mathcal{A}\subseteq \mathcal{B}$, if $A\subseteq B$ and the inclusion map $i: A\to B$ is an embedding.
\end{definition}
\bibliographystyle{amsplain}
\begin{thebibliography}{99}

\end{thebibliography}


\end{document}