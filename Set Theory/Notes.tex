\documentclass[12pt, reqno]{amsart}
\usepackage{amsmath, amsthm, amscd, amsfonts, amssymb, graphicx, xcolor, tikz-cd}
\usepackage[bookmarksnumbered, colorlinks, plainpages]{hyperref}

\textheight 22.5truecm \textwidth 14.5truecm
\setlength{\oddsidemargin}{0.35in}\setlength{\evensidemargin}{0.35in}

\setlength{\topmargin}{-.5cm}

\newtheorem{theorem}{Theorem}[section]
\newtheorem{lemma}[theorem]{Lemma}
\newtheorem{proposition}[theorem]{Proposition}
\newtheorem{corollary}[theorem]{Corollary}
\theoremstyle{definition}
\newtheorem{definition}[theorem]{Definition}
\newtheorem{example}[theorem]{Example}
\newtheorem{exercise}[theorem]{Exercise}
\newtheorem{conclusion}[theorem]{Conclusion}
\newtheorem{conjecture}[theorem]{Conjecture}
\newtheorem{criterion}[theorem]{Criterion}
\newtheorem{summary}[theorem]{Summary}
\newtheorem{axiom}[theorem]{Axiom}
\newtheorem{problem}[theorem]{Problem}
\theoremstyle{remark}
\newtheorem{remark}[theorem]{Remark}
\numberwithin{equation}{section}

\begin{document}
\setcounter{page}{1}

\color{darkgray}{
\noindent 


\centerline{}

\centerline{}


\title[Notes on Set Theory]{Notes on Set Theory}


\author{Xingzhi Huang}

 \maketitle

\section{Preliminaries}

\begin{definition}[Syntax of one-order logic]
    A \textbf{signature} $\sigma$ is a set of relation symbols, function symbols and constant symbols, each with a specified arity. A \textbf{first-order language} $\mathcal{L}$ over a signature $\sigma$ consists of:
    
    1. a countable set of variables $x_0, x_1, x_2, \ldots$
    
    2. the symbols in $\sigma$
    
    3. the logical symbols $\neg, \land, \lor, \to, \leftrightarrow, \forall, \exists, =$
    
    4. the punctuation symbols $(, )$, and $,$.
    
    The terms and formulas of $\mathcal{L}$ are defined inductively as follows:
    
    1. every variable and constant symbol is a term
    
    2. if $f$ is an n-ary function symbol and $t_1, \ldots, t_n$ are terms, then $f(t_1, \ldots, t_n)$ is a term
    
    3. if $R$ is an n-ary relation symbol and $t_1, \ldots, t_n$ are terms, then $R(t_1, \ldots, t_n)$ is an atomic formula
    
    4. if $\varphi$ and $\psi$ are formulas, then so are $\neg \varphi$, $(\varphi \land \psi)$, $(\varphi \lor \psi)$, $(\varphi \to \psi)$ and $(\varphi \leftrightarrow \psi)$
    
    5. if $\varphi$ is a formula and $x$ is a variable, then $\forall x \varphi$ and $\exists x \varphi$ are formulas.
    
    A variable $x$ is free in a formula $\varphi$ if it is not bound by a quantifier in $\varphi$. A formula with no free variables is called a sentence.
\end{definition}

\section{Axioms}

\begin{axiom}[Axiom of Extensionality]
    Two sets are equal if and only if they have the same elements:

    \[
        \forall A \forall B (\forall x (x\in A \iff x\in B) \iff A = B).   
    \]
\end{axiom}

\begin{axiom}[Axiom of Empty Set]
    There exists a set with no elements, denoted by $\emptyset$.

    \[
        \exists A \forall x (x\notin A).
    \]
\end{axiom}

\begin{axiom}[Axiom of Pairing]
    For any sets $A$ and $B$, there exists a set $\{A, B\}$ whose elements are exactly $A$ and $B$.

    \[
        \forall A \forall B \exists C \forall x (x\in C \iff (x = A \lor x = B)).
    \]
\end{axiom}

\begin{axiom}[Axiom of Union]
    For any sets $A$ and $B$, there exists a set $A\cup B$ whose elements are exactly the elements of $A$ or $B$.

    \[
        \forall A \forall B \exists C \forall x (x\in C \iff (x\in A \lor x\in B)).
    \]
\end{axiom}

\begin{axiom}[Axiom of Power Set]
    For any set $A$, there exists a set $\mathcal{P}(A)$ whose elements are exactly the subsets of $A$.

    \[
        \forall A \exists B \forall x (x\in B \iff x\subseteq A).
    \]
\end{axiom}

\begin{axiom}[Subset Axiom Schema/Axiom of Separation]
    For any set $A$ and any formula $\varphi(x,t_0,\ldots,t_n)$ not involving $A$, there exists a set whose elements are exactly the elements of $A$ that satisfy $\varphi(x,t_0,\ldots,t_n)$.

    \[
        \forall t_0, \ldots, t_n \forall A \exists B \forall x (x\in B \iff (x\in A \land \varphi(x,t_0,\ldots,t_n))).
    \]
\end{axiom}

\begin{remark}[Intersection and Union]
    We can define the intersection/union of a set of sets $A$ as follows(Not strictly):
    \[
        \bigcap_{B \in A} B := \{x: \forall B \in A (x \in B)\}, \quad \bigcup_{B \in A} B := \{x: \exists B \in A (x \in B)\}.
    \]
\end{remark}

\begin{remark}[Function]
    A function $f$ from a set $A$ to a set $B$, denoted by $f: A \to B$, is a subset of $A\times B$ such that for every $a\in A$, there exists a unique $b\in B$ such that $(a, b)\in f$. The set $A$ is called the domain of $f$ and the set $B$ is called the codomain of $f$. The element $b$ is called the image of $a$ under $f$, denoted by $f(a)$.
\end{remark}

\begin{axiom}[Axiom of Choice]
    For any relation $R$, there exists a function $f$ such that for every $x$, if there exists a $y$ where $dom(f) = dom(R)$ and $ran(f) \subseteq ran(R)$, and $(x, f(x))\in R$.
\end{axiom}

\section{Natural Numbers}

\begin{definition}
    Let $S$ be a set, we define the successor of $S$ as $S^+ = S \cup \{S\}$.
\end{definition}
\begin{remark}
    Denote $0 = \emptyset$, $1 = 0^+ = \{\emptyset\}$, $2 = 1^+ = \{\emptyset, \{\emptyset\}\}$, $3 = 2^+ = \{\emptyset, \{\emptyset\}, \{\emptyset, \{\emptyset\}\}\}$, and so on.
\end{remark}

\begin{definition}[Inductive Set]
    A set $A$ is called inductive if $\emptyset \in A$ and for every $x\in A$, $x^+ \in A$.
\end{definition}

\begin{axiom}[Axiom of Infinity]
    There exists an inductive set.  
    \[
        \exists A (\emptyset \in A \land \forall x (x\in A \to x^+ \in A)).
    \]
\end{axiom}

\begin{theorem}
    There exists a set $\omega$ consisting of exactly the natural numbers.
\end{theorem}

\begin{proof}
    Suppose $A$ is an inductive set. We use the subset axiom instead of the intersection because the set of all inductive sets may not exist. Let
    \[
        \omega = \{x\in A: \forall B (B \text{ is inductive} \to x\in B)\}.
    \]

    We can show that every element in $\omega$ is either $\emptyset$ or the successor of an element in $\omega$. Thus, $\omega$ consists of exactly the natural numbers.
\end{proof}

\begin{theorem}
    The set $\omega$ is the smallest inductive set, i.e., $\omega$ is an inductive set and for any inductive set $A$, $\omega \subseteq A$.
\end{theorem}

\begin{definition}
    A set $A$ is called transitive if it satisfies the following equivalent conditions:
    
    1. for every $x\in A$, if $y\in x$, then $y\in A$;

    2. $A \subseteq \mathcal{P}(A)$;

    3. $\bigcup A \subseteq A$.
\end{definition}

\begin{theorem}
    Every natural number is a transitive set.
\end{theorem}

\section{Arithmetic of Natural Numbers}

\begin{theorem}
Let $A$ be a set, $F: A \to A$, $a\in A$. There exists a unique function $f: \omega \to A$ such that.
\[
    f(0) = a, \quad f(n^+) = F(f(n))
\]
for every $n\in \omega$.
\end{theorem}

\begin{corollary}
    Let $\mathbb{N}$ be the set of natural numbers with $e \in \mathbb{N}$. There exist unique functions $\sigma: \mathbb{N} \to \mathbb{N}$ such that:

    1. $\sigma(0) = e$;

    2. $\sigma(n^+) = (\sigma(n))^+$ for every $n\in \mathbb{N}$.
\end{corollary}

\begin{definition}
    We can define $A_m: \mathbb{N} \to \mathbb{N}$ as follows:
    \[
        A_m(0) = m, \quad A_m(n^+) = (A_m(n))^+
    \]
\end{definition}

\begin{definition}
    For $m, n \in \mathbb{N}$, we define the addition of $m$ and $n$ as:
    \[
        m + n = A_m(n).
    \]

    where $+(-,-): \mathbb{N} \times \mathbb{N} \to \mathbb{N}$ is a function defined by recursion.
\end{definition}

\begin{theorem}
    The addition on $\mathbb{N}$ satisfies the following properties:

    1. $m + 0 = m$ for every $m\in \mathbb{N}$;

    2. $m + n^+ = (m + n)^+$ for every $m, n \in \mathbb{N}$;

    3. $m + n = n + m$ for every $m, n \in \mathbb{N}$;

    4. $(m + n) + k = m + (n + k)$ for every $m, n, k \in \mathbb{N}$.
\end{theorem}

\begin{definition}
    We can define $M_m: \mathbb{N} \to \mathbb{N}$ as follows:
    \[
        M_m(0) = 0, \quad M_m(n^+) = M_m(n) + m
    \]
\end{definition}

\begin{theorem}
    The multiplication on $\mathbb{N}$ satisfies the following properties:

    1. $m \cdot 0 = 0$ for every $m\in \mathbb{N}$;

    2. $m \cdot n^+ = m \cdot n + m$ for every $m, n \in \mathbb{N}$;

    3. $m \cdot n = n \cdot m$ for every $m, n \in \mathbb{N}$;

    4. $(m \cdot n) \cdot k = m \cdot (n \cdot k)$ for every $m, n, k \in \mathbb{N}$;

    5. $m \cdot (n + k) = m \cdot n + m \cdot k$ for every $m, n, k \in \mathbb{N}$.

\end{theorem}

\section{Integers}

\begin{definition}[Equivalence Relation]
    A relation $R$ on a set $A$ is called an equivalence relation if it satisfies the following properties:

    1. Reflexivity: for every $a\in A$, $(a, a)\in R$;

    2. Symmetry: for every $a, b \in A$, if $(a, b)\in R$, then $(b, a)\in R$;

    3. Transitivity: for every $a, b, c \in A$, if $(a, b)\in R$ and $(b, c)\in R$, then $(a, c)\in R$.
\end{definition}

\begin{definition}[Equivalence Class]
    Let $R$ be an equivalence relation on a set $A$. For every $a\in A$, the equivalence class of $a$ is defined as:
    \[
        [a] = \{b\in A: (a, b)\in R\}.
    \]
\end{definition}

\begin{definition}
    A function $f: A \to B$ where $A$ and $B$ are sets equipped with equivalence relations $R_A$ and $R_B$ respectively is called compatible with the equivalence relations if for every $a_1, a_2 \in A$, if $(a_1, a_2)\in R_A$, then $(f(a_1), f(a_2))\in R_B$.
\end{definition}

\begin{theorem}
    Let $f: A \to B$ be a function compatible with the equivalence relations $R_A$ and $R_B$. Then $f$ induces an unique well-defined function $\bar{f}: A/R_A \to B/R_B$ defined by:
    \[
        \bar{f}([a]) = [f(a)]
    \]
    for every $[a] \in A/R_A$.
\end{theorem}

\begin{definition}[Integers]
    We define the set of integers $\mathbb{Z}$ as the set of equivalence classes of the relation $R$ on $\mathbb{N} \times \mathbb{N}$ defined by:
    \[
        ((a, b), (c, d)) \in R \iff a + d = b + c.
    \]
\end{definition}

\begin{definition}[Addition]
    We define the addition on $\mathbb{Z}$ as follows:
    \[
        [(a, b)] + [(c, d)] = [(a + c, b + d)]
    \]
    for every $[(a, b)], [(c, d)] \in \mathbb{Z}$.
\end{definition}

\begin{proposition}
    The addition on $\mathbb{Z}$ is well-defined and satisfies the following properties:

    1. $x + 0 = x$ for every $x\in \mathbb{Z}$;

    2. $x + y = y + x$ for every $x, y \in \mathbb{Z}$;

    3. $(x + y) + z = x + (y + z)$ for every $x, y, z \in \mathbb{Z}$.

    4. For every $x\in \mathbb{Z}$, there exists $-x \in \mathbb{Z}$ such that $x + (-x) = 0$.
    
    5. $[(0, x)] + [(0,y)] = [(0, x + y)]$ for every $x, y \in \mathbb{N}$.
\end{proposition}



\bibliographystyle{amsplain}
\begin{thebibliography}{99}

\end{thebibliography}


\end{document}